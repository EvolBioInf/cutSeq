\documentclass[a4paper, english]{article}
\usepackage{graphics,eurosym,latexsym}
\usepackage{listings}
\lstset{columns=fixed,basicstyle=\ttfamily,numbers=left,numberstyle=\tiny,stepnumber=5,breaklines=true}
\usepackage{pst-all}
\usepackage{algorithmic,algorithm}
\usepackage{times}
\usepackage{babel}
\usepackage[nodayofweek]{datetime}
\usepackage[round]{natbib}
\bibliographystyle{plainnat}
\oddsidemargin=0cm
\evensidemargin=0cm
\textwidth=16cm
\textheight=23cm
\begin{document}

\title{\texttt{cutSeq} \input{version}: \emph{Description}}
\author{Bernhard Haubold\ignorespaces
}
\input{date}
\date{\displaydate{tagDate}}
\maketitle

\section{Introduction} 
\emph{Introduction}

\section{Getting Started}
The program \texttt{cutSeq} was written in C on a computer running Linux.
Please contact \texttt{haubold@evolbio.mpg.de\ignorespaces
} if there are any problems
with the program.
\begin{itemize}
\item Obtain the package\\
\texttt{git clone https://www.github.com/evolbioinf\ignorespaces
/cutSeq}
\item Change into the directory just downloaded
\begin{verbatim}
cd cutSeq
\end{verbatim}
and make \texttt{cutSeq}
\begin{verbatim}
make
\end{verbatim}
\item Test \texttt{cutSeq}
\begin{verbatim}
make test
\end{verbatim}
\item The executable \texttt{cutSeq} is located in the
  directory \texttt{build}. Place it into your \texttt{PATH}.
\item Make the documentation
\begin{verbatim}
make doc
\end{verbatim}
This calls the typesetting program \texttt{latex}, so please make sure
it is installed before making the documentation. The typeset manual is
located in
\begin{verbatim}
doc/cutSeq.pdf
\end{verbatim}
\end{itemize}

\section{Change Log}
Please use
\begin{verbatim}
git log
\end{verbatim}
to list the change history.

%% \bibliography{ref}
\end{document}

